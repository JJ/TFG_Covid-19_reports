\chapter{Introducción}

Este proyecto es software libre, y está liberado con la licencia \cite{gplv3}.

\section{Motivación}

Vivimos en tiempos extraños, desde diciembre de 2019 escuchamos hablar sobre la aparición de un brote epidemiológico de neumonía cuya causa era desconocida por aquel entonces en la ciudad china de Wuhan. Este brote tuvo lugar en el mercado de la ciudad y tras el aviso de las autoridades competentes de Wuhan a la OMS (Organización Mundial de la Salud) se empezó a investigar que podría estar provocando dicho brote. Esto hizo que se tras varias investigaciones se conociera que la causa del mismo era la aparición de un nuevo coronavirus \cite{oms-covid} de tipo zoonótico, al cual se le dio el nombre de Covid-19. Según declaraciones de la OMS, este virus presentaba un riesgo para la salud pública, bajo las regulaciones del Reglamento Sanitario Internacional \cite{reglamento-sanitario-internacional} y posteriormente este seria considerado como una pandemia.

Poco a poco este nuevo virus fue extendiéndose por el resto del planeta, empezando por el continente asiático y extendiéndose por el resto del planera.Todos y cada uno de los países afectados han aportado los datos de las diferentes incidencias que ha tenido el virus dentro de sus fronteras, aunque a pesar de ello seguimos sin conocer el alcance real de la pandemia ya que no existe un criterio común para la publicación de los datos, si no que cada país los calcula siguiendo los métodos que considera oportunos, por lo que la incidencia puede ser incluso mayor de lo mostrado por los datos, pero no vamos a centrarnos en como se calculan dichos datos.

Basándonos en esto, sabemos que existe una preocupación por parte de la población hacia el virus, ala cual le gusta estar informada. Sabemos que hoy en día es importante estar bien informado, como nos muestra Gabriela Nova en su articulo "Los beneficios de estar informado" \cite{gabriela-nova}. Por ello, la mayor preocupación de la gente es conocer información a cerca del virus, de como está evolucionando, cuantas contagios, muertes y altas se han producido, entre otra información.

Como hemos dicho, a la gente le gusta estar informada, y más ahora, por lo que la transparencia en una situación como la que se está viviendo es algo más que esencial. Esta información tiene como funcionalidad mantener a la gente tranquila, pudiendo aportarles datos de la evolución de la pandemia, pero no solo eso, si no que también sirven para hacer un análisis de cara a la implementación y toma de medidas por parte de los gobiernos de los países afectados. Sin embargo todos estos datos a pesar de ser accesibles para la población, presenta un problema, la difícil visualización de esto de una manera sencilla e intuitiva. En este punto podemos hacernos una pregunta: ¿como podemos ofrecer esa información de manera que pueda ser útil a toda la población que quiera consultar y hacer uso de ella?

\section{Alcance}

\section{Objetivos generales}